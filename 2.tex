\documentclass{article}
\usepackage{homework}

% Title Parameters
\newcommand{\hmwkTitle}{Pset \#1-2}
\newcommand{\hmwkCompletionDate}{\today}
\newcommand{\hmwkClass}{\hspace{6cm} 6.857}
\newcommand{\hmwkClassTime}{Spring 2018}
\newcommand{\hmwkClassRecitation}{}
\newcommand{\hmwkAuthorName}{Austin Garrett, Daniel Prado Sanchez, Elliott Forde}
\newcommand{\hmwkCollaborators}{}

\begin{document}
  \begin{homeworkProblem}
    Re-Using a One-Time Pad \\

    \solution \\

    \part A
      \todo \\

    \part B
      \todo \\

    \part C
      \todo \\

    \part DThe messages are

    \begin{verbatim}
        Cryptography is the study of "mathematical" systems
         involving two kinds of security problems: privacy
        and authentication. A privacy system prevents the e
        xtraction information by unauthorized parties from
        messages transmitted over a public channel, thus as
        suring the sender of a message that it is being rea
        d only by the intended recipient. An authentication
         system prevents the unauthorized injection of mess
        ages into a public channel, assuring the receiver o
        f a message of the legitimacy of its sender.
    \end{verbatim}
    and the pad is 
    \begin{verbatim}
        9e c6 d4 29 00 62 ab 51 7a 72 e5 c1 d4 10 cd d6 17
        54 e4 20 84 50 e4 f9 00 13 fd a6 9f ef 19 d4 60 2a
        42 07 cd d5 a1 01 6d 07 01 32 61 3c 65 9a 8f 5d 33
    \end{verbatim}

    Since we were given ten ciphertexts and know that each character is a readable hex character, we have a bounded number of possible bytes to which each ciphertext byte can decode. So, for each of the 51 bytes, we can iterate through the 256 possible pad bytes and see which byte, when used to decrypt the ciphertext byte, yields readable characters for all ten ciphertexts. The code for this can be found at \url{https://github.com/DanielPradoSanchez/857_PSET_1/blob/master/2d.py}.

  \end{homeworkProblem}
\end{document}
